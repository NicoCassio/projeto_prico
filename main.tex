\documentclass[
	% -- opções da classe memoir --
	12pt,				% tamanho da fonte
	% openright,			% capítulos começam em pág ímpar (insere página vazia caso preciso)
	oneside,			% para impressão em recto e verso. Oposto a oneside
	a4paper,			% tamanho do papel. 
  oldfontcommands,
	% -- opções da classe abntex2 --
	%chapter=TITLE,		% títulos de capítulos convertidos em letras maiúsculas
	%sdsklfjsdfection=TITLE,		% títulos de seções convertidos em letras maiúsculas
	%subsection=TITLE,	% títulos de subseções convertidos em letras maiúsculas
	%subsubsection=TITLE,% títulos de subsubseções convertidos em letras maiúsculas
	% -- opções do pacote babel --
	english,			% idioma adicional para hifenização
	french,				% idioma adicional para hifenização
	spanish,			% idioma adicional para hifenização
	brazil,				% o último idioma é o principal do documento
	]{abntex2}

% ---
% PACOTES
% ---

% ---
% Pacotes fundamentais 
% ---
\usepackage{lmodern}			% Usa a fonte Latin Modern
\usepackage[T1]{fontenc}		% Selecao de codigos de fonte.
\usepackage[utf8]{inputenc}		% Codificacao do documento (conversão automática dos acentos)
\usepackage{indentfirst}		% Indenta o primeiro parágrafo de cada seção.
\usepackage{color}				% Controle das cores
\usepackage{graphicx}			% Inclusão de gráficos
\usepackage{microtype} 			% para melhorias de justificação
% ---

% ---
% Pacotes adicionais, usados apenas no âmbito do Modelo Canônico do abnteX2
% ---
% \usepackage{lipsum}				% para geração de dummy text
% ---

% ---
% Pacotes de citações
% ---
\usepackage[brazilian,hyperpageref]{backref}	 % Paginas com as citações na bibl
\usepackage[alf]{abntex2cite}	% Citações padrão ABNT

% --- 
% CONFIGURAÇÕES DE PACOTES
% --- 

% ---
% Configurações do pacote backref
% Usado sem a opção hyperpageref de backref
\renewcommand{\backrefpagesname}{Citado na(s) página(s):~}
% Texto padrão antes do número das páginas
\renewcommand{\backref}{}
% Define os textos da citação
\renewcommand*{\backrefalt}[4]{
	\ifcase #1 %
		Nenhuma citação no texto.%
	\or
		Citado na página #2.%
	\else
		Citado #1 vezes nas páginas #2.%
	\fi}%
% ---

% ---
% Informações de dados para CAPA e FOLHA DE ROSTO
% ---
\titulo{Modulação de Sinais}
\autor{Nicolas Cassio dos Santos -- SP3037231}
\local{São Paulo}
\data{2024}
\instituicao{%
  Instituto Federal de São Paulo de Educação, Ciência e Tecnologia -- IFSP
  \par
  Campus São Paulo
  \par
  Engenharia Eletrônica}
\tipotrabalho{Projeto prático}
\orientador[Professor:]{Miguel Angelo de Abreu de Sousa}
% O preambulo deve conter o tipo do trabalho, o objetivo, 
% o nome da instituição e a área de concentração 
\preambulo{Projeto de um trabalho teórico e prático a ser realizado para avaliação semestral na discplina de Princípio de Comunicações (PRICO)}
% ---

% ---
% Configurações de aparência do PDF final

% alterando o aspecto da cor azul
\definecolor{blue}{RGB}{41,5,195}

% informações do PDF
\makeatletter
\hypersetup{
     	%pagebackref=true,
		pdftitle={\@title}, 
		pdfauthor={\@author},
    	pdfsubject={\imprimirpreambulo},
	    pdfcreator={LaTeX with abnTeX2},
		pdfkeywords={abnt}{latex}{abntex}{abntex2}{projeto de pesquisa}, 
		colorlinks=true,       		% false: boxed links; true: colored links
    	linkcolor=black,          	% color of internal links
    	citecolor=black,        		% color of links to bibliography
    	filecolor=black,      		% color of file links
		urlcolor=blue,
		bookmarksdepth=4
}
\makeatother
% --- 

% --- 
% Espaçamentos entre linhas e parágrafos 
% --- 

% O tamanho do parágrafo é dado por:
\setlength{\parindent}{1.3cm}

% Controle do espaçamento entre um parágrafo e outro:
\setlength{\parskip}{0.2cm}  % tente também \onelineskip

% ---
% compila o indice
% ---
\makeindex
% ---

\definecolor{blue}{RGB}{0,0,0}

% ----
% Início do documento
% ----
\begin{document}

% Seleciona o idioma do documento (conforme pacotes do babel)
%\selectlanguage{english}
\selectlanguage{brazil}

% Retira espaço extra obsoleto entre as frases.
\frenchspacing 

% ----------------------------------------------------------
% ELEMENTOS PRÉ-TEXTUAIS
% ----------------------------------------------------------
% \pretextual

% ---
% Capa
% ---
\imprimircapa
% ---

% ---
% Folha de rosto
% ---
\imprimirfolhaderosto
% ---

% ---
% NOTA DA ABNT NBR 15287:2011, p. 4:
%  ``Se exigido pela entidade, apresentar os dados curriculares do autor em
%     folha ou página distinta após a folha de rosto.''
% ---

% ---
% inserir lista de ilustrações
% ---
% \pdfbookmark[0]{\listfigurename}{lof}
% \listoffigures*
% \cleardoublepage
% ---

% ---
% inserir lista de tabelas
% ---
% \pdfbookmark[0]{\listtablename}{lot}
% \listoftables*
% \cleardoublepage
% ---

% ---
% inserir lista de abreviaturas e siglas
% ---
% \begin{siglas}
%   \item[ABNT] Associação Brasileira de Normas Técnicas
%   \item[abnTeX] ABsurdas Normas para TeX
% \end{siglas}
% ---

% ---
% inserir lista de símbolos
% ---
% \begin{simbolos}
%   \item[$ \Gamma $] Letra grega Gama
%   \item[$ \Lambda $] Lambda
%   \item[$ \zeta $] Letra grega minúscula zeta
%   \item[$ \in $] Pertence
% \end{simbolos}
% ---

% ---
% inserir o sumario
% ---
\pdfbookmark[0]{\contentsname}{toc}
\tableofcontents*
% \cleardoublepage
% ---


% ----------------------------------------------------------
% ELEMENTOS TEXTUAIS
% ----------------------------------------------------------
\textual

% ----------------------------------------------------------
% Introdução
% ----------------------------------------------------------
\chapter{Introdução}

A comunicação é um aspecto presente na sociedade humana. Antes do surgimento dos estudos em eletricidade, os meios de comunicação possuíam determinadas barreiras de distância e tempo. Com a utilização da comunicação eletrônica, algumas dessas barreiras foram transpostas.

As primeiras técnicas utilizadas para estabelecer esse tipo de comunicação estão dentro do campo da eletrônica analógica. Dentre as teorias e técnicas utlizadas para concretizar esse tipo de comunicação, podem-se destacar a modulação e a demodulação de sinais. Cabe ao engenheiro eletrônico ser capaz de projetar e desenvolver os circuitos responsáveis por essa transformação no sinal de informação.

\section{Objetivos}

O trabalho tem como objetivo apresentar o embasamento teórico necessário para o desenvolvimento de circuitos analógicos moduladores e demoduladores de sinais. Junto a isso, serão de fato desenvolvidos os respectivos circuitos, a \emph{priori} apresentados em um \emph{software} de simulação. Após a simulação, um dos circuitos simulados será apresentado em bancada para a demonstração real do desenvolvimento do trabalho.

\section{Justificativa}

De acordo com o projeto pedagógico do curso (PPC) de bacharelado em engenharia eletrônica \cite{ppc}, a disciplina de PRICO tem como objetivo tornar o aluno competente na análise de sistemas de comunicação análogicos e na avaliação de aspectos técnicos que permeiam o desenvolvimento e a implantação desses sistemas. Acredita-se que durante todo o desenrolar do trabalho proposto, todas as competências citadas serão exploradas e, ao final, poderá ser avaliada a real eficiência atingida.

\section{Metodologia}

A parte teórica do trabalho será desenvolvida através de pesquisa em material bibliográfico que trate sobre os assuntos de comunicação eletrônica, eletrônica analógica, princípios matemáticos utilizados e princípios de eletricidade em geral. Serão utilizados aqueles materiais já citados no PPC \cite{ppc} e também materiais complementares que o autor julgar necessário.

Já para a parte prática, serão utilizados \emph{softwares}, de preferência de código aberto, que possam simular os circuitos desenvolvidos. Posteriormente, técnicas de confecção de placas de circuito impresso serão exploradas para que a montagem do circuito escolhido seja feita.

\section{Cronograma}

O período previsto para a conclusão do trabalho, de forma a se adequar no calendário do semestre letivo \cite{calendario}, foi de 10 semanas. A distribuição das atividades durante esse tempo foi planejado conforme a \autoref{cronograma1} e a \autoref{cronograma2}.

\begin{table}[htb]
\IBGEtab{
  \caption{Cronograma das primeiras semanas.}
  \label{cronograma1}
}{
  \begin{tabular}{c|cccccccccc}
    \toprule
    Etapas & Semana 1 & Semana 2 & Semana 3 & Semana 4 & Semana 5 \\
    \midrule \midrule
    Teoria & X & X &&& \\
    \midrule
    Circuitos &&& X & X & X \\
    \midrule
    Simulação &&&& X & X \\
    \midrule
    Confecção &&&&& \\
    \bottomrule
  \end{tabular}%
}{%
  \fonte{Produzido pelo autor.}%
}
\end{table}

\begin{table}[htb]
\IBGEtab{
  \caption{Cronograma das últimas semanas.}
  \label{cronograma2}
}{
  \begin{tabular}{c|cccccccccc}
    \toprule
    Etapas & Semana 6 & Semana 7 & Semana 8 & Semana 9 & Semana 10 \\
    \midrule \midrule
    Teoria &&&&& \\
    \midrule
    Circuitos X & X & X &&& \\
    \midrule
    Simulação X & X & X & X && \\
    \midrule
    Confecção &&&& X & X \\
    \bottomrule
  \end{tabular}%
}{%
  \fonte{Produzido pelo autor.}%
}
\end{table}

% ----------------------------------------------------------
% Capitulo de textual  
% ----------------------------------------------------------
% \chapter{Desenvolvimento}

% Aqui vai o desenvolvimento.

% ---
% Finaliza a parte no bookmark do PDF
% para que se inicie o bookmark na raiz
% e adiciona espaço de parte no Sumário
% ---
% \phantompart

% ---
% Conclusão
% ---
% \chapter{Considerações finais}

% Aqui talvez seja a conclusão.


% ----------------------------------------------------------
% ELEMENTOS PÓS-TEXTUAIS
% ----------------------------------------------------------
\postextual

% \nocite{livro}

% ----------------------------------------------------------
% Referências bibliográficas
% ----------------------------------------------------------
\bibliography{references}

% ----------------------------------------------------------
% Glossário
% ----------------------------------------------------------
%
% Consulte o manual da classe abntex2 para orientações sobre o glossário.
%
%\glossary

% ----------------------------------------------------------
% Apêndices
% ----------------------------------------------------------

% ---
% Inicia os apêndices
% ---
% \begin{apendicesenv}

% Imprime uma página indicando o início dos apêndices
% \partapendices

% ----------------------------------------------------------
% \chapter{Quisque libero justo}
% ----------------------------------------------------------

% \lipsum[50]

% ----------------------------------------------------------
% \chapter{Nullam elementum urna vel imperdiet sodales elit ipsum pharetra ligula ac pretium ante justo a nulla curabitur tristique arcu eu metus}
% ----------------------------------------------------------
% \lipsum[55-57]

% \end{apendicesenv}
% ---


% ----------------------------------------------------------
% Anexos
% ----------------------------------------------------------

% ---
% Inicia os anexos
% ---
% \begin{anexosenv}

% Imprime uma página indicando o início dos anexos
% \partanexos

% ---
% \chapter{Morbi ultrices rutrum lorem.}
% ---
% \lipsum[30]

% ---
% \chapter{Cras non urna sed feugiat cum sociis natoque penatibus et magnis dis parturient montes nascetur ridiculus mus}
% ---

% \lipsum[31]

% ---
% \chapter{Fusce facilisis lacinia dui}
% ---

% \lipsum[32]

% \end{anexosenv}

%---------------------------------------------------------------------
% INDICE REMISSIVO
%---------------------------------------------------------------------

% \phantompart

% \printindex


\end{document}
